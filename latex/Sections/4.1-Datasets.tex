\subsection{Datasets}

The world is full of graphs and this shows in the available datasets. From citation networks to
online communities to biological networks, all are available online.  We focused on five datasets
for the matrix factorization and random walk based algorithms.

For each dataset we have kept only the largest weakly connected component, as network embeddings
have no meaning between disconnected components.

{\bf \href{https://snap.stanford.edu/data/email-Eu-core.html}{email-EU-core}}: This dataset was
generated using email data from a large European research institution. There is an edge between
nodes $u$ and $v$ if person $u$ sent an email to person $v$.  Each person belongs to one of the 42
departments of the institution. The dataset contains 986 nodes and 25552 edges.

{\bf \href{https://snap.stanford.edu/data/com-Youtube.html}{com-Youtube}~\cite{mislove-2007-socialnetworks}}:
This is the largest dataset we used with 1134890 nodes and 2987624 edges. Edges are friendships and
the classes are user defined groups.

{\bf \href{https://snap.stanford.edu/data/com-DBLP.html}{dblp}}: This is a citation network from
computer science research papers. Every author is a node and two authors are connected if they have
co-authored one paper. Each community is a publication venue where the authors published. The
network has 317080 nodes and 1049866 edges.

{\bf \href{https://snap.stanford.edu/data/com-Amazon.html}{com-amazon}}: This dataset was collected
by crawling the Amazon website. Each node is a product and two products are connected if they appear
in Amazon's {\it Customers Who Bought This Item Also Bought} feature. The groundtruth communities
are the connected components in each product category.  This network has 334863 nodes and 925872
edges.

{\bf \href{http://snap.stanford.edu/graphsage/}{PPI}~\cite{hamilton2017inductive}}: Protein-Protein
Interaction network. Each node in this network corresponds to a protein and and edge exists between
two proteins if an interaction between them has been recorded. There are 24 disconnected graphs
correcsponding to different human tissues.  The labels come from gene ontology (121)
from~\cite{subramanian2005gene}. The average graph has 2360 nodes and average degree 14.4.

{\bf \href{https://github.com/tkipf/gcn/tree/master/gcn/data}{Citeseer, Pubmed, Cora}~\cite{kipf2016semi}}: We also consider three citation network datasets: Citeseer, Cora and Pubmed (Sen et al., 2008). The datasets contain sparse bag-of-words feature vectors for each document and a list of citation links between documents. We treat the citation links as (undirected) edges and construct a binary, symmetric adjacency matrix A. Each document has a class label. For training, we only use 20 labels per class, but all feature vectors.
