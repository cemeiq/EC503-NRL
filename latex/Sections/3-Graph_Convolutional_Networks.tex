\section{Graph Convolutional Networks}

The recent popularity of convolutional neural networks \cite{40k}, and their
success in classifying images and other tasks with a striking accuracy, has intrigued the scientific
community with the question of whether -at least a variant of them- can be leveraged in learning tasks
of graph networked data. The use of graph structures in computer
vision\cite{survey}, as well in the representation of social networks
\cite{kleinberg_book}, has made this task look appealing and worth of investigation-research.
However, using convolutional neural networks for
networked data directly, is not straightforward and it poses significant
challenges.\\
\spara{Challenges in Graph Convolutional Networks} First and foremost,
as CNNs were mainly used for image classification, they assume that
data lie on a regular grid in a geometric space (most commonly
Euclidean). On the contrary, graph data are a typical form of unordered data that lie in
an irregular domain. Also, the heavy-tailed distribution of node degrees
in real networks \cite{smth} makes the filtering-convolution part difficult,
as it is not easy to define a constant-sized neighborhood and apply a
localized filter, as in the traditional CNN case. Also, the pooling stage has to
be defined as well.
\spara{First efforts} First efforts to address the above mentioned issues
by Bruna et al. \cite{Lecun} and introduce the architecture of CNNs to networked data,
mainly draw from the field of Graph Signal Processing \cite{shuman}.
Graphs are generally considered undirected and their representation is given
through their \textbf{Laplacian $L = D -A$} or more commonly the normalized
\textbf{Laplacian $L = I_n - D^{-\frac{1}{2}}AD^{-\frac{1}{2}}$}. As this matrix
is a real symmetric and positive semidefinite, it admits an eigenvector
decomposition $L=U\Lambda U^T$, where $U$ is the matrix of eigenvectors order by
their eigenvalue, and $\Lambda$ is the diagonal matrix of eigenvalues.
