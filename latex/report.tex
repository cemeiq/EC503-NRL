\documentclass{article} % For LaTeX2e
\usepackage{iclr2019_conference,times}

% Optional math commands from https://github.com/goodfeli/dlbook_notation.
\input{math_commands.tex}

\usepackage{hyperref}
\usepackage{url}
\usepackage{graphicx}
\usepackage{booktabs}
\usepackage{multirow}
\usepackage{caption}
\usepackage{colortbl}

\graphicspath{{./figures/}}

\title{A Comparative Evaluation on Network \\Representation Learning Techniques}

% Authors must not appear in the submitted version. They should be hidden
% as long as the \iclrfinalcopy macro remains commented out below.
% Non-anonymous submissions will be rejected without review.

\author{Konstantinos Ameranis \And Panagiota Kiourti \And Konstantinos Sotiropoulos \AND Erasmo Tani \And Isidora Tourni}

% The \author macro works with any number of authors. There are two commands
% used to separate the names and addresses of multiple authors: \And and \AND.
%
% Using \And between authors leaves it to \LaTeX{} to determine where to break
% the lines. Using \AND forces a linebreak at that point. So, if \LaTeX{}
% puts 3 of 4 authors names on the first line, and the last on the second
% line, try using \AND instead of \And before the third author name.

\newcommand{\fix}{\marginpar{FIX}}
\newcommand{\new}{\marginpar{NEW}}

\iclrfinalcopy % Uncomment for camera-ready version, but NOT for submission.
\begin{document}


\maketitle

\begin{abstract}
In this project, we investigate the effectiveness of different techniques to extract features from graph data.
We consider matrix methods, as well as algorithms based on random walks and graph convolutional neural networks. We evaluate the performance of these methods for classification based on ground-truth communities. We found that the performance of each method varies depending on the data considered and no \emph{silver bullet} solution exists. The code, presentation slides, and relevant resources can be found at \url{https://github.com/kameranis/EC503-NRL}.

\end{abstract}

\section{Introduction}
Graph data is pervasive in modern data analysis. In Computer Science, graphs have been an object of study for decades, representing a fundamental tool in the area of data structures and algorithms, and a model for systems ranging from computer networks and social networks to the World Wide Web. The impact of graphs has however vastly exceeded the boundaries of Computer Science, as these models have been used in fields as diverse as Biology, where they represent, for instance protein interaction networks, and Operation Research, in which they appear as models infrastructure networks, such as systems of roads or pipes.

In order to analyze network data it is often useful to obtain a representation of the vertices of the graph in Euclidean Space (an \textbf{embedding}). This allows us to leverage the power of machine learning and data mining techniques that are tailored to Euclidean setting and use them to make sense of the properties of the graph. In order for this to work, the embedding has to somehow encode some of the information and patterns that were contained in the original network. Some desirable properties the embedding should satisfy are the following (\cite{chen2018tutorial}):

\textbf{Adaptability} Since data might be changing over time, the embedding should be efficiently updatable,\\
\textbf{Scalability} The embedding should be computable on large graphs,\\
\textbf{Community Awareness} The Euclidean distance in the embedding should reflect the community structure of the underlying graph,\\
\textbf{Low Dimensionality} Having a low dimensional embedding often allows for better generalization.\\


The problem of finding the best such embedding has proved to be far from trivial, and while many solutions have been proposed, there appears to be no universal answer.

In this project we will look at some of the embedding techniques that have been developed and test their effectiveness in the context of a simple machine learning task.


\subsection{Definitions}
We now now introduce our objects of study.

A \textbf{(undirected) graph} is a pair $(V,E)$, where $V$ is a finite set of \textbf{vertices} (/nodes) and $E$ is a collection of \textbf{edges} (/links). Each element of $E$ is a two-element subset $\{v,u\}$ of $V$, encoding the property that the vertex $v \in V$ is connected to the vertex $u \in V$.\\

A graph \textbf{embedding} for a given graph $G = (V,E)$ is a function $\Phi:V \to \mathbb{R}^{|V| \times d}$.

\section{Network Embeddings}

\subsection{Matrix Factorization}
\subsubsection*{Multidimensional Scaling (MDS)}
Multidimensional Scaling (MDS) is a method for creating a Euclidean embedding of data for which one has distance / dissimilarity information. For instance, given an $N \times N$ matrix of distances between $N$ points, one can embed the points into $\mathbb{R}^k$ so as to preserve distance information. In particular, one can use MDS as a way to create useful features for graphs by considering the shortest path distance between vertices. MDS is similar to PCA, except instead of using correlation information, we make use of pointwise distances.

Classical Multidimensional Scaling works as follows: let $D$ be the dissimilarity matrix. Then:
\begin{enumerate}
  \item Let $D^{(2)}$ be the point-wise square of the distance matrix,
  \item Let $J = I - {1\over n}\vec{1}\vec{1}^T$,
  \item Let $B = -{1 \over 2}JD^{(2)}J$,
  \item Find the top $m$ eigenvalues of $B$ $\lambda_1, ... \lambda_m$, and the corresponding eigenvalues $e_1, ... , e_m$,
  \item Let $X = E_m\Lambda^{1/2}$.
\end{enumerate}
(note to  self:doesn't work if distance not euclidean)


Classical Multidimensional Scaling minimizes a loss function called \emph{strain}:
\[
    Strain_D(x_1, ... , x_N) = \left( ({\sum_{i,j}b_{i,j}- \langle x_i,x_j \rangle})^2 \over \sum_{i,j}b_{i,j}^2\right).
\]

\subsubsection*{Spectral Embedding}
Another way to embed a graph in Euclidean space is given by the spectral embedding. This method computes the $k$ eigenvectors of the normalized Laplacian matrix $\mathcal{L}$  corresponding to the $k$ smallest eigenvalues, and uses each of them as an embedding of the  vertices into $\mathbb{R}$, resulting in an  embedding into $\mathbb{R}^k$. The normalized Laplacian matrix is given by:
\[
    \mathcal{L} = D^{-{1\over 2}}(D - A)D^{-{1\over 2}}
\]

Where $D$ is the $n \times n$ diagonal matrix of degrees of vertices and $A$ is the graph adjacency matrix. One can prove that the quadratic form of the Laplacian is a relaxation to the minimum conductance cut problem (see for instance \ref{chung1997spectral}) defined as follows:

\[
    \underset{S  \subseteq V }{\text{minimize}} {|E(S,\overline{S})|\over \min\{vol(S), vol(\overline{S})\}}.
\]
The eigenvectors of the Laplacian therefore act as optimizers of the relaxation, and tend to align points in space so as to keep connected points close to each other. Note that eigenvectors of the normalized Laplacian are solutions to the generalized eigenvalue problem:
\[
    L\mathbf{x} = \lambda D\mathbf{x}
\]
where $L$ is the (non-normalized) Laplacian matrix: $L = D-A$ and $D$ is defined as above. Spectral embeddings based on the Laplacian are common primitives used in other methods, such as manifold learning, (see for instance \ref{})

\subsubsection*{Isomap}
Isomap is a method for manifold learning / non-linear dimensionality reduction. In a general setting, given data points living in a (non-linear) manifold in $\mathbb{R}^n$ we would like to embed them into lower dimensional space while preserving geodesic distances.

\subsection{Random Walks}

A \emph{random walk}, a term first introduced in~\cite{pearson1905problem},
is a sequence of elements or a path on a mathematical space
produced by a random process. Among other applications, random walks are used
for sampling large graphs of social networks.
At a high level, a random walk in a graph is a sequence of nodes. This sequence
can be thought of as a sentence where each `node' is a `word'. Therefore, this
analogy lets us directly use techniques of word embeddings, for obtaining
embeddings of the nodes. In the next two sections, we introduce
two algorithms that use random walks to learn embeddings of the nodes
by applying an approach that was introduced for learning word embeddings.

\subsubsection{DeepWalk}

DeepWalk is an algorithm that takes a graph as an input, and learns a latent
representation of the nodes. This representation is a vector with continuous
values that aims to capture the social similarity between the nodes while having
a low dimension. In order to learn this representation, DeepWalk uses random
walks and assumes that these walks can capture well the similarity between
nodes. First, it samples a node $v_i$ uniformly at random, from the set of nodes
$V$, as the root of the random walk $W_{v_i}$. Then, it repeatedly samples
a neighbor of the current node uniformly at random, until the walk reaches a
specific length. DeepWalk produces $\gamma$ such random
walks, and then adapts the word2vec approach, which is an approach for obtaining
word embeddings, by treating the random walks as sentences, and the nodes as
words.

The \emph{word2vec} method, introduced in \cite{mikolov2013efficient}, embeds
words of a corpus in a subspace $\mathbb{R}^d$, so that each word embedding can
be used to predict the context around this word, i.e. the `nearby' words (within
a window $w$) in a sentence. In particular, for every word in the vocabulary we
want to calculate its probability of being within a window $w$ from the current
word (Skip-gram method).
In order to do that, Skip-gram parses a given corpus and creates pairs of words
$(w_i, w_j)$ to specify that $w_j$ is within a window $w$ from the word $w_i$.
There is an underlying assumption that the words that are `close' in a
sentence, correspond to the same context. These pairs $(w_i, w_j)$ can then be
used as a training set, to train a model (for example a neural network) to
predict the `nearby' words with high probability from a given word, based on the
training set, i.e. to maximize the likelihood of the training set. These words
are internally represented in a subspace $R^d$. Therefore, the key idea is that
the training of this model yields an embedding for each word in $\mathbb{R}^d$.
Using this embedding we can represent each word as a vector with dimension $d$
which captures the context similarity based on a training set instead of using
an one-hot vector of dimension equal to the size of the vocabulary. The parameter
$d$ is set by the user and it can be tuned using a validation set.

DeepWalk adapts the idea of word embeddings to network graphs by treating the
nodes as words and the random walks as sentences. Similarly to the word
embeddings, the goal is for the embeddings to capture node similarity so that
they can then be used for classifying them. There is no rigorous justification
about why random walks can capture the information of community-similar nodes of
a social network graph. However, the authors observe that the word frequencies in
some corpora, such as the English Wikipedia, follows a power-law distribution,
which is also the case for the degree of nodes in social networks. That
observation possibly makes the random walks suitable for capturing the
neighborhood similarity or community membership in the social network graphs.

DeepWalk first initializes a mapping
$\Phi: V \to \mathbb{R}^{\mid V\mid \times d}$ which maps each node to an
embedding of dimension $d$. Then, the objective is to solve the following
optimization problem:
\begin{align}
    \min_{\Phi}
        \sum_{W_{u}: u \in V} \sum_{v_i \in W_{v_i}} \left (-\log{Pr({v_{i-w},
                \ldots, v_{i-1}, v_{i+1},
                \ldots , v_{i+w}} \mid \Phi(v_i))}
        \right )
\end{align}
where $v_i$ is the node currently examined in the random walk $W_{v_i}$,
$\Phi(v_i)$ is the embedding of this node from which we want to predict the
nearby nodes and $v_{i-w}, \ldots, v_{i-1}, v_{i+1}, \ldots, v_{i+w}$ are the
nearby nodes (within a window $w$ from the node $v_i$) in this walk.
$\Phi$ is updated using gradient descent. In the end, DeepWalk outputs the
matrix $\Phi$, which corresponds to the final node embedding.

\subsubsection{Node2vec}

Similarly to DeepWalk, \emph{Node2vec} (\cite{grover2016node2vec}) is a method
for learning latent representations of the nodes by using random walks and the
skip-gram. However, this method uses a different way of deciding the next node
to be sampled during a random walk. The reason is that there are graphs that
exhibit in order to handle graphs that have
structural equivalent nodes differently than the ones that include nodes
highly connected which exhibit community similarity.

Node2vec introduces two baselines cases for sampling neighbor nodes: the
Breadth-first Sampling (BFS) and the Depth-first Sampling (DFS).
The Breadth-first strategy samples a neighborhood for a node $v_i$ by
considering as neighbors, the nodes that have an immediate edge to $v_i$.
In contrast, Depth-first strategy considers as neighbors the nodes that are
one more edge away from $v_i$ each time we sample for one more node in the
neighborhood. As an example, consider the node $u$ in
Fig~\ref{sampling_strategies} (obtained by the original
paper~\cite{grover2016node2vec}). Sampling a neighborhood of size 3 using the
BFS strategy can give us the nodes $s_1, s_2, s_3$, while sampling using the DFS
strategy can give us the nodes $s_4, s_5, s_6$.
\begin{figure}
\begin{center}
\includegraphics[width=0.7\textwidth]{figures/sampling.png}
\end{center}
\caption{Using the BFS sampling strategy, we can obtain the nodes
$s_1, s_2, s_3$ as the 3 neighbors of $u$. Using the DFS sampling strategy we
can obtain $\{s_4, s_5, s_6\}$ as a neighborhood of size 3. The figure is copied
from the original paper~\cite{grover2016node2vec}.}
\label{sampling_strategies}
\end{figure}
The DFS strategy samples nodes in a way that explores neighborhoods of nodes. 
Thus, it can find nodes that are structurally equivalent. For example, consider 
a network graph with nodes that correspond to hubs or bridges. In this case the 
embedding will reflect the structural equivalence of the nodes of the graph.
On the contrary, the BFS strategy can create random walks consisting of nearby 
nodes that are highly connected. This means that the nodes belong to similar 
clusters or communities and exhibit \emph{homophily}.

Real-world graph networks include both structural similar nodes and as well as
homophilous nodes that exhibit community similarities by being highly connected.
The node2vec approach can capture both similarities by introducing the parameters
$p$ and $q$. The return parameter $p$ controls the likelihood of returning back
to the previous node in the walk. The parameter $q$ controls the decision
between a BFS and a DFS sampling strategy. If $q$ is set to be less than 1
then the walk is more biased to choose nodes that are further from the current
node in the walk. If $q$ is chosen to be more than 1, then the walk would choose
nodes that are close to the current node with higher probability, following a
strategy closer to the BFS strategy. Formally, the algorithm uses the transition
probabilities $\pi$ to sample the next node in the walk. For example, the
transition probabilities for sampling the node $x$ when being in the node $u$ are:
\begin{align}
\pi_{ux} = \begin{cases} \frac{1}{p}\cdot w_{ux} & \text{ if } d_{tx} = 0 \\
1\cdot w_{ux} & \text{ if } d_{tx} = 1 \\
\frac{1}{q}\cdot w_{ux} & \text{ if } d_{tx} = 2
\end{cases}
\end{align}
where $t$ is the previous node in the random walk as shown in
Fig~\ref{node2vec_png} and $d_{tx}$ is the shortest path from $t$ to $x$.
\begin{figure}
\begin{center}
\includegraphics[width=0.5\textwidth]{figures/node2vec.png}
\end{center}
\caption{A random walk in the node2vec algorithm. The current node is the node
$v$ which was chosen as a neighbor to the vertex $t$.
Going back to $t$ is controlled by the parameter $p$, while going forward is
controlled by parameter $q$. The figure is copied from the original
paper~\cite{grover2016node2vec}.}
\label{node2vec_png}
\end{figure}

\subsubsection{LINE}

LINE (\cite{DBLP:journals/corr/TangQWZYM15}) is a network embedding model, able to handle very large, arbitrary types of graphs (undirected, directed and/or weighted), by optimizing an objective which preserves both local and global network structures. Local structures are represented by the observed links in the networks, since for each pair of vertices linked by edge $(u, v)$, the weight $w_{uv}$ on that edge indicates the first-order proximity between $u$ and $v$. Global structure is the second-order proximity between the vertices, determined through the shared neighborhood structures. This corresponds to the intuitive notion that nodes with shared neighbors are likely to be similar. If $p_u = (w_{u,1}; \dots ;w_{u,|V|})$ denotes the first-order proximity of $u$ with all the other vertices, then the second-order proximity between $u$ and $v$ is the similarity between $p_u$ and $p_v$.

In both cases, an objective function is defined and optimized, and the difference in the variants of the model that are described lies on whether first- or second-order proximity (or both) is used, and on the method selected for optimizing this objective in each case. Using first-order proximity, the KL-divergence between the joint and the empirical distribution of vertices $v_i$ and $v_j$ for each undirected edge $(i,j)$ is minimized. On the other hand, using second-order proximity, $u_i'$ is defined as the representation of $v_i$, when it is treated as a specific "context", and $u_i$ as the representation of $v_i$ when it is treated as a vertex. In this case, the KL-divergence between the conditional and the empirical distribution over the contexts is minimized.

A negative sampling approach tackles the problem of trivial infinity solutions in the case of first-order proximity and that of the computationally expensive minimization of the objective in the case of second-order proximity. Multiple negative edges are sampled according to some noisy distribution for each edge $(i,j)$, and asynchronous Stochastic Gradient Descent algorithm is used, which, in each step, samples a mini-batch of edges and updates the model parameters.
When edge weights have a high variance, scales of the gradients in SGD diverge, making it harder to find a good learning rate. Optimization via edge-sampling is then applied, by sampling from the original edges, with sampling probabilities proportional to the original edge weights. Sampled edges are treated as binary edges.

A simple way to combine first- and second-order proximity information, is to concatenate the vector representations learned by both approaches into a longer vector, and reweigh the dimensions to balance the two representations. It was found that in practice, optimization takes $O(|E|)$ time, and the overall complexity of LINE is $O(d(K+1))$, given that $K$ is the number of negative samples and $d<<|V|$ the dimension of the lower-dimensional space.

As a way to better combine first- and second-order proximities, the authors propose to jointly train the objective function in the future. Also, they suggest that higher-order proximity approaches could be applied to provide a better result. Another objective could be to find new embeddings of heterogeneous information networks, meaning graphs with vertices of multiple types. Furthermore, the case of no observed connections between new and existing vertices could be explored in the future by resorting to other network information, such as the textual information of the vertices.

\subsubsection{HARP}

HARP (\cite{DBLP:journals/corr/ChenPHS17}) is a meta-strategy which solves the graph representation learning problem using a hierarchical approach. All of the method previously described in this section often get stuck at a bad local minimum as the result of poor initialization of the non-convex optimization. Moreover, these methods mostly aim to preserve local proximities in a graph but neglect its global structure.

HARP recursively coalesces the nodes and edges in the original graph to get a series of successively smaller but structurally similar graphs . These coalesced graphs, each with a different granularity, provide a view of the original graph’s global structure. Starting from the most simplified form, each graph is used to learn a set of initial representations which serve as good initializations for embedding the next, more detailed graph. This process is repeated until we get an embedding for each node in the original graph.

HARP's method for multi-level graph representation learning consists of three parts:
\begin{itemize}
\item \textbf{Graph Coarsening:} starting from the original graph, $G = G_0$, a hierarchy of successively smaller graphs $G_0, G_1, \dots , G_L$ is created. A hybrid graph coarsening scheme is developed, which is repeatedly applied to obtain a small graph. It combines two algorithms, edge collapsing and star collapsing, preserving first- and second-order proximity respectively. Edge collapsing is an edge selection and node merging algorithm, which arbitrarily merges nodes of edges into single nodes, constructing a graph with at least half the edges of the original one. Star collapsing algorithm, on the other hand, considers star-like structured graphs, and merges nodes with the same neighbors into supernodes.
     \item \textbf{Graph Embedding:}  Using a provided Graph Embedding algorithm, Graph Embedding is obtained on the Coarsest Graph $G_L$, a small sized graph providing a high quality representation.
     \item \textbf{Representation Prolongation and Refinement:} For each graph $G_i$, the graph representation of $G_{i+1}$ is prolonged and taken as its initial embedding, $\Phi'_{G_i}$, followed by applying a provided embedding algorithm to $(G_i, \Phi'_{G_i})$ to further refine $\Phi'_{G_i}$ and obtain refined embedding $\Phi_{G_i}$.
\end{itemize}

Processes of Graph Embedding, Prolongation and Refinement are then applied recursively to the larger graphs and their embeddings, until we obtain the graph embedding of the original graph, $\Phi'_{G_0}$.

HARP is combined with a few state-of-the-art graph embedding methods (DeepWalk, LINE, Node2vec) to produce higher quality embeddings for all of them.
Time Complexity of HARP(DW) is the same as that of the original DeepWalk, equal to $O(\gamma |V|tw (d+dlog|V|))$, where $\gamma$ is the number of random walks, t is the  walk length, w is the window size, d is the representation size, and $|V|$ is the number of nodes. Similarly, time complexity of HARP(LINE) is the same as that of LINE, $O(r|E|)$,  linear to the number of edges in the graph, $|E|$, and the number of iterations over edges, r.


\section{Graph Convolutional Networks}

The recent popularity of convolutional neural networks \cite{40k}, and their
success in classifying images and other tasks with a striking accuracy, has intrigued the scientific
community with the question of whether -at least a variant of them- can be leveraged in learning tasks
of graph networked data. The use of graph structures in computer
vision\cite{survey}, as well in the representation of social networks
\cite{kleinberg_book}, has made this task look appealing and worth of investigation-research.
However, using convolutional neural networks for
networked data directly, is not straightforward and it poses significant
challenges.\\
\spara{Challenges in Graph Convolutional Networks} First and foremost,
as CNNs were mainly used for image classification, they assume that
data lie on a regular grid in a geometric space (most commonly
Euclidean). On the contrary, graph data are a typical form of unordered data that lie in
an irregular domain. Also, the heavy-tailed distribution of node degrees
in real networks \cite{smth} makes the filtering-convolution part difficult,
as it is not easy to define a constant-sized neighborhood and apply a
localized filter, as in the traditional CNN case. Also, the pooling stage has to
be defined as well.
\spara{Convolution in graphs} First efforts to address the above mentioned issues
by Bruna et al. \cite{Lecun} and introduce the architecture of CNNs to networked data,
mainly draw from the field of Graph Signal Processing \cite{shuman}.
Graphs are generally considered undirected and their representation is given
through their \textbf{Laplacian $L = D -A$} or more commonly the normalized
\textbf{Laplacian $L = I_n - D^{-\frac{1}{2}}AD^{-\frac{1}{2}}$}. As this matrix
is a real symmetric and positive semidefinite, it admits an eigenvector
decomposition $L=U\Lambda U^T$, where $U$ is a square orthonormal matrix with
the eigenvectors as its columns, and $\Lambda$ is the diagonal matrix of eigenvalues.
Letting $x\in \mathbb{R}^{n}$ be a feature vector of the nodes of a graph,
the {\em graph fourier transform} is then defined as $\hat{x}=U^T x \in \mathbb{R}^n$
and its inverse as $x = U\hat{x}$. The \textbf{convolution operator} on a graph
$\mathcal{G}$ is defined on the Fourier domain, as:
\begin{equation*}
x *_{\mathcal{G}} y = U((U^T x)\odot (U^T y))
\end{equation*}
where $\odot$ denotes the element-wise Hadamard product.\\
Therefore, a signal $x$ is filtered by $g_{\theta}$, as:\\
\begin{equation}
y = g_{\theta}(L)x = g_{\theta} (U\Lambda U^T)x = U g_{\theta}(\Lambda ) U^T x
\end{equation}
where $g_{\theta}(\Lambda) = diag(\theta)$ is a non-parametric filter.\\
This approach has, however, the following limitations:
\begin{itemize}
\item [1.] Filters are not localized
\item [2.] Their learning complexity scales with the dimensionality of the data $O(n)$
\item [3.] The computational cost of filtering is high- $O(n^2)$, due to the
multiplication with the Fourier basis $U$.
\end{itemize}

\subsubsection*{ChebNet\cite{defferard}}

\section{Comparative Evaluation}

\subsection{Datasets}

The world is full of graphs and this shows in the available datasets. From citation networks to
online communities to biological networks, all are available online.  We focused on five datasets
for the matrix factorization and random walk based algorithms.

For each dataset we have kept only the largest weakly connected component, as network embeddings
have no meaning between disconnected components.

{\bf \href{https://snap.stanford.edu/data/email-Eu-core.html}{email-EU-core}}: This dataset was
generated using email data from a large European research institution. There is an edge between
nodes $u$ and $v$ if person $u$ sent an email to person $v$.  Each person belongs to one of the 42
departments of the institution. The dataset contains 986 nodes and 25552 edges.

{\bf \href{https://snap.stanford.edu/data/com-Youtube.html}{com-Youtube}~\cite{mislove-2007-socialnetworks}}:
This is the largest dataset we used with 1134890 nodes and 2987624 edges. Edges are friendships and
the classes are user defined groups.

{\bf \href{https://snap.stanford.edu/data/com-DBLP.html}{dblp}}: This is a citation network from
computer science research papers. Every author is a node and two authors are connected if they have
co-authored one paper. Each community is a publication venue where the authors published. The
network has 317080 nodes and 1049866 edges.

{\bf \href{https://snap.stanford.edu/data/com-Amazon.html}{com-amazon}}: This dataset was collected
by crawling the Amazon website. Each node is a product and two products are connected if they appear
in Amazon's {\it Customers Who Bought This Item Also Bought} feature. The groundtruth communities
are the connected components in each product category.  This network has 334863 nodes and 925872
edges.

{\bf \href{http://snap.stanford.edu/graphsage/}{PPI}~\cite{hamilton2017inductive}}: Protein-Protein
Interaction network. Each node in this network corresponds to a protein and and edge exists between
two proteins if an interaction between them has been recorded. There are 24 disconnected graphs
correcsponding to different human tissues.  The labels come from gene ontology (121)
from~\cite{subramanian2005gene}. The average graph has 2360 nodes and average degree 14.4.

\subsection{Evaluation Tasks}

The task that we chose to evaluate the embedding algorithms on is node classification. Specifically,
we run Logisitic Regression on each groundtruth label. After that we report micro and macro
precision, recall and F1 score, as well as the weighted F1 score.

\subsection{Comparative Evaluation}

Unfortunately not all algorithms could be run on all the datasets. MDS and Spectral Embedding
require quadratic space with respect to the number of the nodes, giving them an effective ceiling of
about 10000 nodes. The largest dataset, com-Youtube proved too much for all implementations except
deepwalk. We were unable to run the published code for LINE, depsite our best efforts.

The following table~\ref{tab:results} shows the micro and macro precision, recall and F1 score for
classification in each dataset.

\begin{table}[h]
    \captionof{table}{Classification results on different embedding algorithms and datasets}\label{tab:results}
\centerline{
\begin{tabular}{ll|rrr|rrr|r}
\toprule
    &                   &  \multicolumn{3}{c|}{micro} &  \multicolumn{3}{c|}{macro} & weighted \\
    &                   &  Precision & Recall & F1 &  Precision &  Recall &  F1 &  F1 \\
\midrule
\multirow{5}{*}{email-EU-core} & deepwalk & \cellcolor{yellow!75} 0.9550 & \cellcolor{yellow!75} 0.9550 & \cellcolor{yellow!75} 0.9550 & \cellcolor{yellow!75} 0.9181 & \cellcolor{yellow!75} 0.9264 & \cellcolor{yellow!75} 0.9144 & \cellcolor{yellow!75} 0.9482 \\
 & node2vec & 0.6113 & 0.6113 & 0.6113 & 0.3065 & 0.3561 & 0.3035 & 0.5507 \\
 & HARP & 0.6923 & 0.6923 & 0.6923 & 0.5181 & 0.5296 & 0.4901 & 0.6535 \\
 & MDS & 0.0324 & 0.0324 & 0.0324 & 0.0454 & 0.0441 & 0.0278 & 0.0291 \\
 & SpectralEmbedding & 0.6640 & 0.6640 & 0.6640 & 0.5976 & 0.5797 & 0.5467 & 0.6706 \\
\cline{1-9}
com-Youtube & deepwalk & 0.9029 & 0.9029 & 0.9029 & 0.5650 & 0.9117 & 0.5674 & 0.9426 \\
\cline{1-9}
\multirow{3}{*}{dblp} & deepwalk & 0.9457 & 0.9457 & 0.9457 & 0.6299 & 0.8954 & 0.6683 & 0.9610 \\
 & node2vec & 0.9557 & 0.9557 & 0.9557 & 0.6373 & \cellcolor{yellow!75} 0.9376 & 0.6879 & 0.9677 \\
 & HARP & \cellcolor{yellow!75} 0.9565 & \cellcolor{yellow!75} 0.9565 & \cellcolor{yellow!75} 0.9565 & \cellcolor{yellow!75} 0.6521 & 0.9040 & \cellcolor{yellow!75} 0.6964 & \cellcolor{yellow!75} 0.9679 \\
\cline{1-9}
\multirow{3}{*}{com-amazon} & deepwalk & 0.9985 & 0.9985 & 0.9985 & \cellcolor{yellow!75} 0.9829 & 0.9854 & 0.9828 & 0.9985 \\
 & node2vec & 0.9980 & 0.9980 & 0.9980 & 0.9615 & \cellcolor{yellow!75} 0.9944 & 0.9750 & 0.9981 \\
 & HARP & \cellcolor{yellow!75} 0.9987 & \cellcolor{yellow!75} 0.9987 & \cellcolor{yellow!75} 0.9987 & 0.9814 & 0.9913 & \cellcolor{yellow!75} 0.9847 & \cellcolor{yellow!75} 0.9987 \\
\cline{1-9}
\multirow{5}{*}{PPI} & deepwalk & 0.6617 & 0.6617 & 0.6617 & 0.6088 & 0.6409 & 0.6060 & 0.6821 \\
 & node2vec & 0.6540 & 0.6540 & 0.6540 & 0.6079 & 0.6410 & 0.6022 & 0.6761 \\
 & HARP & 0.6610 & 0.6610 & 0.6610 & 0.6100 & \cellcolor{yellow!75}0.6427 & 0.6064 & 0.6819 \\
 & MDS & 0.5294 & 0.5294 & 0.5294 & 0.5005 & 0.5009 & 0.4770 & 0.5621 \\
 & SpectralEmbedding & \cellcolor{yellow!75} 0.6911 & \cellcolor{yellow!75} 0.6911 & \cellcolor{yellow!75} 0.6911 & \cellcolor{yellow!75} 0.6178 & 0.6419 & \cellcolor{yellow!75} 0.6211 & \cellcolor{yellow!75} 0.7040 \\
\bottomrule
\end{tabular}
}
\end{table}


The only method that lags behind every other algorithm by a lot. Deepwalk remains always very close
to the best algorithm. In email-EU-core all algorithms other than deepwalk perform horribly
comparably. HARP which utilizes deepwalk has the best performance in the dblp dataset. It is
surprising that while Spectral Embedding is so old, in the PPI dataset outperforms all other
methods. The authors believe that this is because the PPI graphs have very small cuts that
correspond to the the cuts between clusters of interacting proteins.

As stated at the start of this report, no single algorithm can produce embeddings superior to all
others for all algorithms. Even a single algorithm is usually highly customizable with many
parameter choices whose optimums change depending on the input graph.



\section{Conclusions}

\subsubsection*{Other directions}


Most of the aforementioned context searching strategies used in network embedding models rely on a definition of context nodes that is identical for all networks and does not adapt to the properties or the application domain of each graph. Therefore, much work has been done on unifying different network embeddings under a general framework. The most significant ones are the following:
\begin{itemize}
\item \textbf{GraphAttention:}
In \cite{velivckovic2017graph}, the proposed method automatically learns different attention parameters for different networks, by parameterizing the attention over the power series of a transition matrix.
\item \textbf{GEM-D:}
In \cite{chen2017fast}, the authors' approach decomposes graph embedding algorithms, such as Laplacian Eigenvectors, DeepWalk, LINE, and node2vec, into three building blocks: node proximity function, warping function and loss function. It shows that these algorithms can all be unified under this framework, and tests different design choices for each building block on real-world graphs, to pick the triple which works the best empirically.  
\item \textbf{NetMF:}
In \cite{Qiu:2018:NEM:3159652.3159706}, the authors show that models such as DeepWalk, LINE, PTE, and node2vec, which use a negative sampling method, can be unified into a matrix factorization framework with closed forms, and provide the theoretical connections between skip-gram based network embedding algorithms and graph Laplacian. Based on these observations, they present the NetMF method for DeepWalk and LINE, as well as its approximation algorithm for computing network embedding, and show it offers significant improvements over the aforementioned models for conventional network mining tasks.
\end{itemize}

Similarly, many methods have been proposed to reduce the dependence of the existing embedding methods upon general loss functions
and optimization models, which are not tuned for any particular task, and hence tend to have a suboptimal performance in comparison to end-to-end task-specific embedding methods. 
For example, in \cite{abu2017learning}, the authors propose a new definition of the graph likelihood function, designed specifically for link prediction.


\section{Work Distribution}
\begin{itemize}
\item Konstaninos Ameranis: Code for downloading and cleaning datasets. Code for running algorithms
on datasets and classification on produced embeddings. Section about datasets, tasks and
results.
\item Panagiota Kiourti: code for installing and running DeepWalk, node2vec, HARP. Sections explaining random walks and the following methods: DeepWalk, word2vec, 
node2vec.
\item Konstantinos Sotiropoulos: Worked on Graph Neural Networks: Explored and chose the papers from the literature, designed experiments, implemented and modified code, run the experiments, and wrote the relevant section. 
\item Erasmo Tani: Worked on the theory part of the MDS and Spectral Embedding, wrote the corresponding section, the introduction and the abstract.
\item Isidora Tourni: Performed the theoretical analysis of LINE, HARP, wrote the corresponding sections, and also contributed to the introduction sections and graph neural networks' results section. Also contributed to the HARP code parameterization.
\end{itemize}



\bibliography{iclr2019_conference}
\bibliographystyle{iclr2019_conference}

\end{document}
